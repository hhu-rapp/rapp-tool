\subsection{Cost Complexity Pruning}

\begin{figure}[ht]
  \centering
  \begin{tikzpicture}
    \begin{axis}[
      xlabel=Baumtiefe,
      ylabel=Balanced Accuracy,
      legend style={anchor=south west, at={(0.04,0.04)}}
    ]
      \addplot+[only marks] coordinates {
        {{#nonpareto_coords}}
        ({{depth}}, {{performance}})
        {{/nonpareto_coords}}
      };
      \addplot+[only marks] coordinates {
        {{#pareto_coords}}
        ({{depth}}, {{performance}}) % alpha={{alpha}}
        {{/pareto_coords}}
      };
      \legend{,Paretofront}

    \end{axis}
  \end{tikzpicture}
  \caption{Übersicht der verschiedenen Baumtiefen und Performance-Werte
    im CCP für verschiedene \(\alpha\)-Werte. Die Paretofront ist
    hervorgehoben.}
\end{figure}

{{#pareto_front}}
\clearpage
\subsubsection{ {{title}} }
{{#figure}}
\Cref{fig:{{label}}-tree} zeigt den trainierten Baum.
{{/figure}}
Die Klassifikations-Performance und die Fairness-Ergebnisse sind in
\cref{tab:{{label}}-performance,tab:{{label}}-fairness}
abgebildet

{{#figure}}
\begin{figure}[ht]
  {{=<< >>=}}
  \includegraphics[width=\linewidth]{<<figure>>}
  <<={{ }}=>>
  \caption{Trainierter Entscheidungsbaum für {{titel}}.}
  \label{fig:{{label}}-tree}
\end{figure}
{{/figure}}


{{{performance_table}}}
{{{fairness_table}}}

{{/pareto_front}}
